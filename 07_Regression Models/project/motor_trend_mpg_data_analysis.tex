\documentclass[]{article}
\usepackage{lmodern}
\usepackage{amssymb,amsmath}
\usepackage{ifxetex,ifluatex}
\usepackage{fixltx2e} % provides \textsubscript
\ifnum 0\ifxetex 1\fi\ifluatex 1\fi=0 % if pdftex
  \usepackage[T1]{fontenc}
  \usepackage[utf8]{inputenc}
\else % if luatex or xelatex
  \ifxetex
    \usepackage{mathspec}
  \else
    \usepackage{fontspec}
  \fi
  \defaultfontfeatures{Ligatures=TeX,Scale=MatchLowercase}
\fi
% use upquote if available, for straight quotes in verbatim environments
\IfFileExists{upquote.sty}{\usepackage{upquote}}{}
% use microtype if available
\IfFileExists{microtype.sty}{%
\usepackage{microtype}
\UseMicrotypeSet[protrusion]{basicmath} % disable protrusion for tt fonts
}{}
\usepackage[margin=1in]{geometry}
\usepackage{hyperref}
\hypersetup{unicode=true,
            pdftitle={Motor Trend MPG Data Analysis},
            pdfauthor={Ram Mohan Rao Adduri},
            pdfborder={0 0 0},
            breaklinks=true}
\urlstyle{same}  % don't use monospace font for urls
\usepackage{color}
\usepackage{fancyvrb}
\newcommand{\VerbBar}{|}
\newcommand{\VERB}{\Verb[commandchars=\\\{\}]}
\DefineVerbatimEnvironment{Highlighting}{Verbatim}{commandchars=\\\{\}}
% Add ',fontsize=\small' for more characters per line
\usepackage{framed}
\definecolor{shadecolor}{RGB}{248,248,248}
\newenvironment{Shaded}{\begin{snugshade}}{\end{snugshade}}
\newcommand{\AlertTok}[1]{\textcolor[rgb]{0.94,0.16,0.16}{#1}}
\newcommand{\AnnotationTok}[1]{\textcolor[rgb]{0.56,0.35,0.01}{\textbf{\textit{#1}}}}
\newcommand{\AttributeTok}[1]{\textcolor[rgb]{0.77,0.63,0.00}{#1}}
\newcommand{\BaseNTok}[1]{\textcolor[rgb]{0.00,0.00,0.81}{#1}}
\newcommand{\BuiltInTok}[1]{#1}
\newcommand{\CharTok}[1]{\textcolor[rgb]{0.31,0.60,0.02}{#1}}
\newcommand{\CommentTok}[1]{\textcolor[rgb]{0.56,0.35,0.01}{\textit{#1}}}
\newcommand{\CommentVarTok}[1]{\textcolor[rgb]{0.56,0.35,0.01}{\textbf{\textit{#1}}}}
\newcommand{\ConstantTok}[1]{\textcolor[rgb]{0.00,0.00,0.00}{#1}}
\newcommand{\ControlFlowTok}[1]{\textcolor[rgb]{0.13,0.29,0.53}{\textbf{#1}}}
\newcommand{\DataTypeTok}[1]{\textcolor[rgb]{0.13,0.29,0.53}{#1}}
\newcommand{\DecValTok}[1]{\textcolor[rgb]{0.00,0.00,0.81}{#1}}
\newcommand{\DocumentationTok}[1]{\textcolor[rgb]{0.56,0.35,0.01}{\textbf{\textit{#1}}}}
\newcommand{\ErrorTok}[1]{\textcolor[rgb]{0.64,0.00,0.00}{\textbf{#1}}}
\newcommand{\ExtensionTok}[1]{#1}
\newcommand{\FloatTok}[1]{\textcolor[rgb]{0.00,0.00,0.81}{#1}}
\newcommand{\FunctionTok}[1]{\textcolor[rgb]{0.00,0.00,0.00}{#1}}
\newcommand{\ImportTok}[1]{#1}
\newcommand{\InformationTok}[1]{\textcolor[rgb]{0.56,0.35,0.01}{\textbf{\textit{#1}}}}
\newcommand{\KeywordTok}[1]{\textcolor[rgb]{0.13,0.29,0.53}{\textbf{#1}}}
\newcommand{\NormalTok}[1]{#1}
\newcommand{\OperatorTok}[1]{\textcolor[rgb]{0.81,0.36,0.00}{\textbf{#1}}}
\newcommand{\OtherTok}[1]{\textcolor[rgb]{0.56,0.35,0.01}{#1}}
\newcommand{\PreprocessorTok}[1]{\textcolor[rgb]{0.56,0.35,0.01}{\textit{#1}}}
\newcommand{\RegionMarkerTok}[1]{#1}
\newcommand{\SpecialCharTok}[1]{\textcolor[rgb]{0.00,0.00,0.00}{#1}}
\newcommand{\SpecialStringTok}[1]{\textcolor[rgb]{0.31,0.60,0.02}{#1}}
\newcommand{\StringTok}[1]{\textcolor[rgb]{0.31,0.60,0.02}{#1}}
\newcommand{\VariableTok}[1]{\textcolor[rgb]{0.00,0.00,0.00}{#1}}
\newcommand{\VerbatimStringTok}[1]{\textcolor[rgb]{0.31,0.60,0.02}{#1}}
\newcommand{\WarningTok}[1]{\textcolor[rgb]{0.56,0.35,0.01}{\textbf{\textit{#1}}}}
\usepackage{graphicx,grffile}
\makeatletter
\def\maxwidth{\ifdim\Gin@nat@width>\linewidth\linewidth\else\Gin@nat@width\fi}
\def\maxheight{\ifdim\Gin@nat@height>\textheight\textheight\else\Gin@nat@height\fi}
\makeatother
% Scale images if necessary, so that they will not overflow the page
% margins by default, and it is still possible to overwrite the defaults
% using explicit options in \includegraphics[width, height, ...]{}
\setkeys{Gin}{width=\maxwidth,height=\maxheight,keepaspectratio}
\IfFileExists{parskip.sty}{%
\usepackage{parskip}
}{% else
\setlength{\parindent}{0pt}
\setlength{\parskip}{6pt plus 2pt minus 1pt}
}
\setlength{\emergencystretch}{3em}  % prevent overfull lines
\providecommand{\tightlist}{%
  \setlength{\itemsep}{0pt}\setlength{\parskip}{0pt}}
\setcounter{secnumdepth}{0}
% Redefines (sub)paragraphs to behave more like sections
\ifx\paragraph\undefined\else
\let\oldparagraph\paragraph
\renewcommand{\paragraph}[1]{\oldparagraph{#1}\mbox{}}
\fi
\ifx\subparagraph\undefined\else
\let\oldsubparagraph\subparagraph
\renewcommand{\subparagraph}[1]{\oldsubparagraph{#1}\mbox{}}
\fi

%%% Use protect on footnotes to avoid problems with footnotes in titles
\let\rmarkdownfootnote\footnote%
\def\footnote{\protect\rmarkdownfootnote}

%%% Change title format to be more compact
\usepackage{titling}

% Create subtitle command for use in maketitle
\providecommand{\subtitle}[1]{
  \posttitle{
    \begin{center}\large#1\end{center}
    }
}

\setlength{\droptitle}{-2em}

  \title{Motor Trend MPG Data Analysis}
    \pretitle{\vspace{\droptitle}\centering\huge}
  \posttitle{\par}
    \author{Ram Mohan Rao Adduri}
    \preauthor{\centering\large\emph}
  \postauthor{\par}
      \predate{\centering\large\emph}
  \postdate{\par}
    \date{May 31, 2019}


\begin{document}
\maketitle

\emph{Created with knitr}

\hypertarget{executive-summary}{%
\paragraph{Executive Summary}\label{executive-summary}}

This report analyzed the relationship between transmission type (manual
or automatic) and miles per gallon (MPG). The report set out to
determine which transmission type produces a higher MPG. The
\texttt{mtcars} dataset was used for this analysis. A t-test between
automatic and manual transmission vehicles shows that manual
transmission vehicles have a 7.245 greater MPG than automatic
transmission vehicles. After fitting multiple linear regressions,
analysis showed that the manual transmission contributed less
significantly to MPG, only an improvement of 1.81 MPG. Other variables,
weight, horsepower, and number of cylinders contributed more
significantly to the overall MPG of vehicles.

\hypertarget{load-data}{%
\paragraph{Load Data}\label{load-data}}

Load the dataset and convert categorical variables to factors.

\begin{Shaded}
\begin{Highlighting}[]
\KeywordTok{library}\NormalTok{(ggplot2)}
\KeywordTok{data}\NormalTok{(mtcars)}
\KeywordTok{head}\NormalTok{(mtcars, }\DataTypeTok{n=}\DecValTok{3}\NormalTok{)}
\KeywordTok{dim}\NormalTok{(mtcars)}
\NormalTok{mtcars}\OperatorTok{$}\NormalTok{cyl <-}\StringTok{ }\KeywordTok{as.factor}\NormalTok{(mtcars}\OperatorTok{$}\NormalTok{cyl)}
\NormalTok{mtcars}\OperatorTok{$}\NormalTok{vs <-}\StringTok{ }\KeywordTok{as.factor}\NormalTok{(mtcars}\OperatorTok{$}\NormalTok{vs)}
\NormalTok{mtcars}\OperatorTok{$}\NormalTok{am <-}\StringTok{ }\KeywordTok{factor}\NormalTok{(mtcars}\OperatorTok{$}\NormalTok{am)}
\NormalTok{mtcars}\OperatorTok{$}\NormalTok{gear <-}\StringTok{ }\KeywordTok{factor}\NormalTok{(mtcars}\OperatorTok{$}\NormalTok{gear)}
\NormalTok{mtcars}\OperatorTok{$}\NormalTok{carb <-}\StringTok{ }\KeywordTok{factor}\NormalTok{(mtcars}\OperatorTok{$}\NormalTok{carb)}
\KeywordTok{attach}\NormalTok{(mtcars)}
\end{Highlighting}
\end{Shaded}

\hypertarget{exploratory-analysis}{%
\paragraph{Exploratory Analysis}\label{exploratory-analysis}}

\textbf{See Appendix Figure I} Exploratory Box graph that compares
Automatic and Manual transmission MPG. The graph leads us to believe
that there is a significant increase in MPG when for vehicles with a
manual transmission vs automatic.

\hypertarget{statistical-inference}{%
\subparagraph{Statistical Inference}\label{statistical-inference}}

T-Test transmission type and MPG

\begin{Shaded}
\begin{Highlighting}[]
\NormalTok{testResults <-}\StringTok{ }\KeywordTok{t.test}\NormalTok{(mpg }\OperatorTok{~}\StringTok{ }\NormalTok{am)}
\NormalTok{testResults}\OperatorTok{$}\NormalTok{p.value}
\end{Highlighting}
\end{Shaded}

\begin{verbatim}
## [1] 0.001373638
\end{verbatim}

The T-Test rejects the null hypothesis that the difference between
transmission types is 0.

\begin{Shaded}
\begin{Highlighting}[]
\NormalTok{testResults}\OperatorTok{$}\NormalTok{estimate}
\end{Highlighting}
\end{Shaded}

\begin{verbatim}
## mean in group 0 mean in group 1 
##        17.14737        24.39231
\end{verbatim}

The difference estimate between the 2 transmissions is 7.24494 MPG in
favor of manual.

\hypertarget{regression-analysis}{%
\subparagraph{Regression Analysis}\label{regression-analysis}}

Fit the full model of the data

\begin{Shaded}
\begin{Highlighting}[]
\NormalTok{fullModelFit <-}\StringTok{ }\KeywordTok{lm}\NormalTok{(mpg }\OperatorTok{~}\StringTok{ }\NormalTok{., }\DataTypeTok{data =}\NormalTok{ mtcars)}
\KeywordTok{summary}\NormalTok{(fullModelFit)  }\CommentTok{# results hidden}
\KeywordTok{summary}\NormalTok{(fullModelFit)}\OperatorTok{$}\NormalTok{coeff  }\CommentTok{# results hidden}
\end{Highlighting}
\end{Shaded}

Since none of the coefficients have a p-value less than 0.05 we cannot
conclude which variables are more statistically significant.

Backward selection to determine which variables are most statistically
significant

\begin{Shaded}
\begin{Highlighting}[]
\NormalTok{stepFit <-}\StringTok{ }\KeywordTok{step}\NormalTok{(fullModelFit)}
\KeywordTok{summary}\NormalTok{(stepFit) }\CommentTok{# results hidden}
\KeywordTok{summary}\NormalTok{(stepFit)}\OperatorTok{$}\NormalTok{coeff }\CommentTok{# results hidden}
\end{Highlighting}
\end{Shaded}

The new model has 4 variables (cylinders, horsepower, weight,
transmission). The R-squared value of 0.8659 confirms that this model
explains about 87\% of the variance in MPG. The p-values also are
statistically significantly because they have a p-value less than 0.05.
The coefficients conclude that increasing the number of cylinders from 4
to 6 with decrease the MPG by 3.03. Further increasing the cylinders to
8 with decrease the MPG by 2.16. Increasing the horsepower is decreases
MPG 3.21 for every 100 horsepower. Weight decreases the MPG by 2.5 for
each 1000 lbs increase. A Manual transmission improves the MPG by 1.81.

\hypertarget{residuals-diagnostics}{%
\paragraph{Residuals \& Diagnostics}\label{residuals-diagnostics}}

Residual Plot \textbf{See Appendix Figure II}

The plots conclude:

\begin{enumerate}
\def\labelenumi{\arabic{enumi}.}
\tightlist
\item
  The randomness of the Residuals vs.~Fitted plot supports the
  assumption of independence
\item
  The points of the Normal Q-Q plot following closely to the line
  conclude that the distribution of residuals is normal
\item
  The Scale-Location plot random distribution confirms the constant
  variance assumption
\item
  Since all points are within the 0.05 lines, the Residuals vs.~Leverage
  concludes that there are no outliers
\end{enumerate}

\begin{Shaded}
\begin{Highlighting}[]
\KeywordTok{sum}\NormalTok{((}\KeywordTok{abs}\NormalTok{(}\KeywordTok{dfbetas}\NormalTok{(stepFit)))}\OperatorTok{>}\DecValTok{1}\NormalTok{)}
\end{Highlighting}
\end{Shaded}

\begin{verbatim}
## [1] 0
\end{verbatim}

\hypertarget{conclusion}{%
\paragraph{Conclusion}\label{conclusion}}

There is a difference in MPG based on transmission type. A manual
transmission will have a slight MPG boost. However, it seems that
weight, horsepower, \& number of cylinders are more statistically
significant when determining MPG.

\hypertarget{appendix-figures}{%
\subsubsection{Appendix Figures}\label{appendix-figures}}

\hypertarget{i}{%
\paragraph{I}\label{i}}

\includegraphics{motor_trend_mpg_data_analysis_files/figure-latex/unnamed-chunk-7-1.pdf}

\hypertarget{ii}{%
\paragraph{II}\label{ii}}

\includegraphics{motor_trend_mpg_data_analysis_files/figure-latex/unnamed-chunk-8-1.pdf}


\end{document}
